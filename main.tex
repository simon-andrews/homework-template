\documentclass{article}

\usepackage{amsmath,amsthm,amssymb,amsfonts} % math symbols
\usepackage{enumitem} % used for setting default enumerate label to letters e.g. (a), (b), etc.
\usepackage{fullpage} % make margins smaller
\usepackage{graphicx} % for embedding images
\usepackage[pdf]{graphviz} % for pretty graphs
\usepackage{hyperref} % clickable and styled links
\usepackage[utf8]{inputenc} % enable non-ASCII characters
\usepackage{minted} % code highlighting
\usepackage{multirow} % enable table cells spanning multiple rows

% set default enumerate label to letters
\setlist[enumerate]{label={(\alph*)}}

% environment for problems
\newenvironment{problem}[1]
{ \begin{trivlist} \item[\hskip \labelsep {\bfseries\Large Problem #1}] \item }
{ \end{trivlist} }

% environment for solutions (goes inside problems)
\newenvironment{solution}
{ \begin{trivlist} \item[\hskip \labelsep {\bfseries\large Solution}] \item }
{ \end{trivlist} \newpage }

% environment for solution parts for multi-part problems (goes inside solutions)
\newenvironment{solutionpart}[1]
{ \begin{trivlist} \item[\hskip \labelsep {\bfseries Part #1}] \item }
{ \end{trivlist} }

\title{SUBJECT 101: Homework \#0}
\author{John Q. Public}
\date{Due \today}

\begin{document}

\maketitle

This is a sample of Simon Andrews's homework template for \LaTeX. This part of the file is a space for you to put whatever you want, such as acknowledgements or notes to the instructors. In this sample, I'll be putting a quick demo for how to use this template.

For every problem in the problem set, create a new .tex file and use the \textbackslash input command to include its contents in this main.tex file. Each problem file should have this structure:

\begin{minted}{latex}
% Problem statement
\begin{problem}{N.N.N}
Jane has 2 apples and Jack has 3 apples. How many apples do both of them have
\begin{enumerate}
    \item if Jane eats 1 apple, and
    \item if Jack eats 2 apples?
\end{enumerate}
\end{problem}

% Solution
\begin{solution}

% Common stuff useful for all problems.
Jane and Jack together have 5 apples.

% Solution for part A
\begin{solutionpart}{A}
If Jane eats 1 apple, Jane and Jack together have $5 - 1 = 4$ apples.
\end{solutionpart}

% Solution for part B
\begin{solutionpart}{B}
If Jack eats 2 apples, Jane and Jack together have $5 - 2 = 3$ apples.
\end{solutionpart}

\end{solution}
\end{minted}

\newpage

% Place your problem files here
\begin{problem}{1.1.1}
Jane has 2 apples and Jack has 3 apples. How many apples do both of them have
\begin{enumerate}
    \item if Jane eats 1 apple, and
    \item if Jack eats 2 apples?
\end{enumerate}
\end{problem}

\begin{solution}

Jane and Jack together have 5 apples.

\begin{solutionpart}{A}
If Jane eats 1 apple, Jane and Jack together have $5 - 1 = 4$ apples.
\end{solutionpart}

\begin{solutionpart}{B}
If Jack eats 2 apples, Jane and Jack together have $5 - 2 = 3$ apples.
\end{solutionpart}

\end{solution}

\end{document}
